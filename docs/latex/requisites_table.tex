% Tabella dei requisiti
\begin{table}[h]
    \definecolor{prioritygreen}{RGB}{106, 168, 79}    % Darker green
    \definecolor{priorityyellow}{RGB}{241, 194, 50}   % Darker yellow/gold
    \definecolor{priorityred}{RGB}{224, 102, 102}     % Darker red
    \definecolor{headercolor}{RGB}{79, 98, 120}       % Dark blue-gray
    % Cella di identificativo della colonna
    \newcommand{\headingcell}[1]{\multicolumn{1}{|c|}{#1}}
    % Riga contenente le informazioni di un requisito
    \newcommand{\requisiteline}[4] {
        #1 & #2 & #3 &
        \multicolumn{1}{c|}{\IfStrEqCase{#4}{
            {Bassa}{\cellcolor{prioritygreen}}
            {Media}{\cellcolor{priorityyellow}}
            {Alta}{\cellcolor{priorityred}}
        }[\cellcolor{white}]#4}\\
        \hline
    }
    
	\centering
	\begin{tabularx}{\textwidth}{
            | >{\hsize=.2\hsize\centering\arraybackslash}X
            | >{\hsize=.3\hsize}X
            | >{\hsize=.4\hsize}X
            | >{\hsize=.1\hsize}X |
        }
		\hline
		\rowcolor{headercolor}\multicolumn{4}{|c|}{Lista dei Requisiti} \\
		\hline
        \rowcolor{lightgray}\headingcell{Identificativo} & \headingcell{Nome} & \headingcell{Descrizione} & \headingcell{Priorità}\\
        \hline
		\requisiteline{FR-1}{Input di Studente}{Descrizione di test1}{Media}
        \requisiteline{FR-2}{Test di Studente}{Descrizione di test2}{Alta}
        \requisiteline{FR-3}{Dati di Studente}{Descrizione di test3}{Bassa}
	\end{tabularx}
\end{table}