\documentclass[a4paper, 12pt]{article}
\usepackage[italian]{babel}
\usepackage[table]{xcolor}
\usepackage{xstring, xltabular, multirow, ifthen, hyphenat, svg, tcolorbox, enumitem, adjustbox}
\usepackage[
colorlinks = true,
linkcolor=black,
urlcolor=black,
citecolor=black
]{hyperref}

\hypersetup{linkcolor=black}

% Titolo del documento e informazioni sul lavoro
\title{
	\Large EasyLibrary\\
	\vspace{1mm}
	\Huge Design di Dettaglio del Software
}
% Metodo per centrare correttamente con wrapping i nomi.
\author{
	\textbf{Gruppo 18}\\
	Francesco Cangianiello\\
	Andrea Ciliberti\\
	Serena Giannitti\\
	Marco Giraulo
}

\begin{document}
	
\maketitle

\newpage
\tableofcontents

\newpage
%\section{Architettura del Software}
%\subsectio{Diagramma delle Classi}

\section{Descrizione generale del sistema}
\begin{itemize}
	\item Architettura generale
	\item Principi di progettazione adottati (SOLID, DRY, KISS, ecc.)
	\item Tecnologie e strumenti scelti
\end{itemize}

\section{Progettazione di dettaglio}

\subsection{Diagrammi delle classi}
\begin{itemize}
	\item Descrizione delle classi principali
	\item Inserimento del diagramma UML (SVG):
	\begin{figure}[h!]
		\centering
		%\includesvg[width=0.8\textwidth]{diagrammi/classi.svg}
		\caption{Diagramma delle classi principali}
	\end{figure}
	\item Spiegazione testuale:
	\begin{itemize}
		\item Responsabilità di ciascuna classe
		\item Relazioni (associazioni, aggregazioni, composizioni, ereditarietà)
		\item Motivazione delle scelte progettuali
		\item Riferimento ai principi di progettazione (Single Responsibility, Open/Closed, ecc.)
	\end{itemize}
\end{itemize}

\subsection{Diagrammi di sequenza}
\begin{itemize}
	\item Selezionare casi d'uso complessi
	\item Per ogni caso d'uso:
	\begin{itemize}
		\item Nome del caso d'uso
		\item Attori coinvolti
		\item Diagramma di sequenza (SVG):
		\begin{figure}[h!]
			\centering
			%\includesvg[width=0.8\textwidth]{diagrammi/sequenza_caso1.svg}
			\caption{Diagramma di sequenza per il caso d'uso XYZ}
		\end{figure}
		\item Descrizione testuale passo-passo
		\item Motivazioni progettuali
	\end{itemize}
\end{itemize}


\end{document}

