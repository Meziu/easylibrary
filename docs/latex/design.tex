\documentclass[a4paper, 12pt]{article}
\usepackage[italian]{babel}
\usepackage[table]{xcolor}
\usepackage{xstring, xltabular, multirow, ifthen, hyphenat, svg, tcolorbox, enumitem, adjustbox}
\usepackage[
colorlinks = true,
linkcolor=black,
urlcolor=black,
citecolor=black
]{hyperref}

\hypersetup{linkcolor=black}

% Titolo del documento e informazioni sul lavoro
\title{
	\Large EasyLibrary\\
	\vspace{1mm}
	\Huge Progettazione di Dettaglio del Software
}
% Metodo per centrare correttamente con wrapping i nomi.
\author{
	\textbf{Gruppo 18}\\
	Francesco Cangianiello\\
	Andrea Ciliberti\\
	Serena Giannitti\\
	Marco Giraulo
}

\newcommand{\umlbox}[1]{
	\begin{figure}[ht]
		\begin{adjustbox}{center, margin=0cm 0cm 0cm 0cm}
			#1
		\end{adjustbox}
	\end{figure}
}

\begin{document}
	
\maketitle

\newpage
\tableofcontents

\raggedright

\newpage

\section{Introduzione}
\subsection{Scopo del documento}
Lo scopo di questo di questo documento è di descrivere la progettazione del livello di dettaglio
del il sistema software EasyLibrary. Il documento rappresenta la mappa per
l'implementazione del sistema utilizzata dagli sviluppatori durante il processo di implementazione.
Nel documento sono presenti descrizioni per i principali componenti del sistema EasyLibrary
e esposizioni delle scelte di progettazione che hanno portato alla loro formalizzazione.
I diagrammi di sequenza sono utilizzati per dimostrare il funzionamento dinamico del sistema e per
descriverne con precisione i meccanismi di implementazione agli sviluppatori.

\subsection{Processo di Progettazione}
Completata la fase di analisi, è stato innanzitutto disegnato un diagramma UML generale delle classi
con cui astrarre le caratteristiche del sistema per requisiti. Successivamente, al fine di migliorare
la struttura del sistema secondo gli attributi di qualità studiati, sono state compiute scelte di design
che facendo riferimento alle capacità delle tecnologie di sviluppo (linguaggio Java).
Sfruttando appieno la flessibilità dell'incapsulamento, dell'ereditarietà e dei tipi generici,
si è arrivati ad una struttura del sistema maggiormente stabile e più facilmente manutenibile e testabile.
A tal punto si è potuto passare all finalizzazione dei contratti delle diverse classi e dei loro metodi associati. Infine, i casi d'uso del sistema sono stati passati in rassegna e per ognuno è stato
individuato un flusso di sequenza delle operazioni, da cui sono stati dedotti i diagrammi di
sequenza e di attività. Durante il processo di disegno di questi diagrammi sono stati riconosciuti
e risolti alcuni problemi di design, cosa che ha portato a migliorare in modo iterativo la
progettazione del sistema.


\newpage
\section{Progettazione di Dettaglio delle Classi}
Il sistema software è stato scomposto in 2 pacchetti funzionali:
\begin{itemize}
	\item Un pacchetto responsabile della gestione delle strutture dati che compongono l'archivio della biblioteca.
	\item Un pacchetto responsabile della rappresentazione dei dati elementari dell'archivio della biblioteca.
\end{itemize}

\subsection{Gestori di Archivio}
\umlbox{
	\includesvg[width=1.2\textwidth, inkscapelatex=false]{../uml/classesGestori.svg}
}

\subsection{Rappresentazione Dati}
\umlbox{
	\includesvg[width=1.2\textwidth, inkscapelatex=false]{../uml/classesDati.svg}
}


\newpage
\section{Diagrammi di Sequenza}
\begin{itemize}
	\item Selezionare casi d'uso complessi
	\item Per ogni caso d'uso:
	\begin{itemize}
		\item Nome del caso d'uso
		\item Attori coinvolti
		\item Diagramma di sequenza (SVG):
		\begin{figure}[h!]
			\centering
			%\includesvg[width=0.8\textwidth]{diagrammi/sequenza_caso1.svg}
			\caption{Diagramma di sequenza per il caso d'uso XYZ}
		\end{figure}
		\item Descrizione testuale passo-passo
		\item Motivazioni progettuali
	\end{itemize}
\end{itemize}


\end{document}

