\documentclass[a4paper, 12pt]{article}
\usepackage[italian]{babel}
\usepackage[table]{xcolor}
\usepackage{xstring, xltabular, multirow, ifthen, hyphenat, svg, tcolorbox, enumitem, adjustbox}
\usepackage[
	colorlinks = true,
	linkcolor=black,
	urlcolor=black,
	citecolor=black
]{hyperref}

\hypersetup{linkcolor=black}

% Titolo del documento e informazioni sul lavoro
\title{
    \Large EasyLibrary\\
    \vspace{1mm}
    \Huge Specifica dei Requisiti del Software
}
% Metodo per centrare correttamente con wrapping i nomi.
\author{
    \textbf{Gruppo 18}\\
    Francesco Cangianiello\\
    Andrea Ciliberti\\
    Serena Giannitti\\
    Marco Giraulo
}

\begin{document}

\maketitle

\newpage
\tableofcontents


\newpage
\section{Requisiti di Progetto}
\newcommand{\requisitebox}[8]{%
	\hypertarget{#1}{}
	\begin{tcolorbox}[colback=white,colframe=black!60,rounded corners]
		\textbf{ID:} #1 \\[2pt]
		\textbf{Nome:} #2 \\[2pt]
		\textbf{Descrizione:} #3 \\[2pt]
		\textbf{Input:} #4 \\[2pt]
		\textbf{Output:} #5 \\[2pt]
		\textbf{Precondizioni:} #6 \\[2pt]
		\textbf{Postcondizioni:} #7 \\[2pt]
		\textbf{Priorità:} #8
	\end{tcolorbox}
}


\subsection{Specifica dei Requisiti}
\hypertarget{IF}{\subsubsection{Funzionalità individuali}}
\requisitebox{IF-1.1}{Registrazione di un nuovo utente}{Il sistema consente la registrazione di un nuovo utente previa compilazione dei seguenti dati obbligatori: nome, cognome, \\matricola ed e-mail istituzionale.\\	La matricola deve essere univoca, ovvero non può coincidere con quella di altri utenti già presenti nel sistema.\\
Inoltre, sia la matricola sia l’indirizzo e-mail devono rispettare il\\ formato predefinito (da DF-1.1).\\ Se il gestore inserisce matricola o e-mail istituzionale non conformi l'interazione di aggiunta viene disabilitata}{Nome, cognome, matricola, e-mail istituzionale}{Nessuno.}{Le informazioni
\\sull’utente devono corrispondere alle specifiche del requisito DF-1.1.}{Il nuovo utente viene aggiunto all’elenco degli utenti registrati.}{Alta}
\requisitebox{IF-1.2}{Modifica dei dati di un utente}{Il sistema permette di modificare le generalità fornite dall’utente. Tutte le informazioni possono essere modificate (campi DF-1.1) eccetto i prestiti attivi e la matricola. Se le nuove informazioni non sono valide (matricola e /o indirizzo email non conformi alle specifiche o matricola duplicata nell'archivio del sistema) il campo non viene modificato bensì restituisce un messaggio di errore tornando al valore iniziale.}{Nuovi valori dei campi dell'utente.}{Nessuno.}{L'utente è correttamente registrato nel sistema.}{L'utente selezionato viene aggiornato, e le modifiche sono riflesse nell’archivio.}{Alta}
\requisitebox{IF-1.3}{Rimozione di un utente}{Il sistema permette la cancellazione di un utente \\registrato, purché non abbia prestiti attivi associati al \\proprio profilo. Se invece l’utente possiede uno o più prestiti attivi, il sistema disabiliterà l'interazione della cancellazione.}{Matricola associata all’utente da rimuovere.}{Nessuno.}{Il profilo dell’utente è presente nel sistema.}{L'utente selezionato è rimosso dall'archivio, e le modifiche sono riflesse nell’archivio.}{Alta}
\requisitebox{IF-1.4}{Ricerca di utenti}{Il sistema permette la ricerca di utenti tramite 2 modalità: per cognome o per matricola. Il sistema restituisce tutti gli utenti che soddisfano il criterio di ricerca. Se non esistono nell’archivio del sistema utenti che corrispondono ai criteri di ricerca, sarà restituita una lista vuota di utenti.}{Criteri di ricerca (cognome, matricola).}{Utenti corrispondenti ai criteri di ricerca.}{Il criterio di ricerca è valido.}{Nessuna.}{Media}
\requisitebox{IF-2.1}{Inserimento di un nuovo libro}{Il sistema consente l’aggiunta di un nuovo libro \\all’archivio tramite l’inserimento dei campi necessari alla sua \\rappresentazione (DF-1.2).
L’operazione di inserimento è consentita solo se l’ISBN non risulta già presente nel sistema e rispetta il \\formato definito in DF-1.2.
Inoltre, il numero di copie disponibili del libro inserito non può essere inferiore a zero, e l'anno di pubblicazione non può essere successivo alla data odierna.}{Titolo, lista di autori (nome e cognome), anno di pubblicazione, codice identificativo (ISBN), numero di copie disponibili.}{Nessuno.}{Nessuna.}{Il libro è aggiunto all’archivio.}{Alta}
\requisitebox{IF-2.2}{Modifica dei dati relativi ad un libro}{Il sistema permette di modificare i dati relativi ad un libro (DF-1.2) nel sistema. È possibile modificare le informazioni \\relative a titolo, autore, anno di pubblicazione e numero di copie disponibili di un libro. L’anno di pubblicazione non può essere \\successivo all’anno corrente e il numero di copie disponibili non può essere inferiore a zero. Le modifiche devono mantenere la coerenza dei dati.}{Nuovi dati relativi al libro.}{Nessuno.}{Il libro è presente nel sistema.}{I dati del libro risultano aggiornati secondo le \\modifiche specificate, e le modifiche sono riflesse nell’archivio.}{Alta}
\requisitebox{IF-2.3}{Rimozione di un libro}{Il sistema deve permettere di eliminare un libro \\dall’archivio, cancellandone le informazioni correlate.}{Dati relativi al libro da rimuovere.}{Nessuno.}{Il libro deve essere presente nell’archivio del sistema.}{Il libro è rimosso dall'archivio, e le modifiche sono riflesse nell’archivio.}{Alta}
\requisitebox{IF-2.4}{Ricerca di libri}{Il gestore deve poter cercare libri secondo tre modalità:\vspace{-8pt}
	\begin{itemize}
		\renewcommand{\labelitemi}{-}
		\setlength\itemsep{-0.1cm}
		\item{per titolo}
		\item{per autore}
		\item{per ISBN (codice univoco)}
	\end{itemize}
	 \vspace{-8pt} il sistema deve restituire tutti i libri che soddisfano il criterio di \\ricerca. Se non esistono nell’archivio del sistema libri che \\corrispondono ai criteri di ricerca, sarà restituita una lista vuota di libri.}{Criterio di ricerca (Titolo, autore o ISBN del libro).}{Elenco dei libri corrispondenti ai criteri della ricerca.}{Il criterio di ricerca è valido.}{Nessuna.}{Media}
\requisitebox{IF-3.1}{Raccolta dello storico dei prestiti}{Il gestore deve poter ottenere e operare sull'elenco dei prestiti attivi e restituiti.}{Nessuno.}{Elenco completo di tutti i prestiti (attivi e restituiti) presenti nell'archivio del sistema.}{Nessuna.}{Nessuna.}{Media}
\requisitebox{IF-3.2}{Modifica di un prestito}{Il sistema deve permettere al gestore di modificare la data di scadenza di un prestito già registrato. La nuova data non può essere precedente alla data corrente.}{Nuova data di scadenza del prestito.}{Nessuno.}{Il prestito deve essere correttamente registrato nel sistema.}{La data di scadenza del prestito deve essere modificata e salvata. Le modifiche sono riflesse nell’archivio.}{Bassa}

\hypertarget{BF}{\subsubsection{Business Flow}}
\requisitebox{BF-1.1}{Registrazione di un nuovo prestito}{Il gestore registra il prestito di un libro da parte di un utente. Un prestito può essere registrato solo se: \vspace{-6pt}
	\begin{itemize}
 	    \renewcommand{\labelitemi}{-}
 		\setlength\itemsep{-0.1cm}
		\item{il libro ha almeno una copia disponibile;}
		\item{l’utente ha meno di tre prestiti attivi.}
		\item{la data di scadenza è successiva alla data di registrazione.}
	\end{itemize}
	\vspace{-6pt} Se una o più delle condizioni sopra indicate non risultano soddisfatte, il gestore sarà notificato di quali condizioni non siano rispettate e non potrà confermare l’operazione. Al termine della registrazione, il sistema deve decrementare il numero di copie disponibili del libro e inserire le informazioni del prestito nell’archivio (DF-1.3). \\È possibile per un utente prendere in prestito più di una singola copia del medesimo libro.}{Matricola dell’utente che vuole effettare il prestito e ISBN del libro da prendere in prestito.}{Nessuno.}{L’utente a carico del prestito è correttamente registrato nel sistema. Il libro da prestare è correttamente registrato nel sistema. L'archivio possiede almeno una copia disponibile del libro. L'utente ha meno di 3 prestiti attivi a carico. La data di scadenza è successiva alla data di registrazione.}{Il numero di copie disponibili del libro è decrementato. Il prestito viene aggiunto alla lista dei prestiti attivi dell’utente. Le modifiche sono riflesse nell’archivio.}{Alta}
\requisitebox{BF-1.2}{Restituzione di un libro}{Il gestore registra la restituzione di un libro da parte di un utente. Il gestore ricerca il prestito corrispondente all’interno del sistema e lo segna come restituito. Se tale prestito era già contrassegnato come restituito, il sistema non compierà operazioni aggiuntive. Altrimenti, il sistema incrementa il numero di copie disponibili del libro.}{Dati del libro da restituire e dell’utente che effettua la restituzione.}{Nessuno.}{Il prestito deve essere attivo.}{Il prestito viene segnalato come restituito. La copia del libro viene resa disponibile per nuovi prestiti. Le modifiche sono riflesse nell’archivio.}{Alta}


\hypertarget{DF}{\subsubsection{Esigenze Dati}}
\requisiteboxsimple{DF-1.1}{Informazioni sull’utente}{Un utente deve essere rappresentato tramite i seguenti dati: nome, cognome, matricola univoca (stringa di 10 cifre numeriche), e-mail istituzionale (formattata per standard RFC 5322 semplificato, terminante nel campo di dominio con “studenti.unisa.it”), lista dei titoli dei libri in prestito (DF-1.3) attualmente attivi a carico \\dell’utente, che non possono essere più di 3. \\Nessuno dei campi dei dati può essere vuoto.}{Alta}
\requisiteboxsimple{DF-1.2}{Informazioni sul libro}{Un libro deve essere rappresentato tramite i seguenti dati: titolo, lista di autori (con ognuno campo di nome e di cognome non vuoti), anno di pubblicazione (precedente o corrispondente all'anno attuale), codice identificativo univoco ISBN (per standard ISO 2108) e numero di copie disponibili nell’archivio (non negative). \\Nessuno dei campi dei dati può essere vuoto.}{Alta}
\requisiteboxsimple{DF-1.3}{Informazioni sul prestito}{Un prestito è rappresentato tramite le informazioni di matricola dell'utente a carico (DF-1.1), di codice ISBN del libro preso in prestito (DF-1.2), di data prevista di restituzione del prestito, e di un indicatore che ne indica lo stato tra "attivo" o "restituito".\\Nessuno dei campi dei dati può essere vuoto.}{Alta}
\requisitebox{DF-2.1}{Salvataggio dei dati dell’archivio su file}{Ogni componente dell’archivio (libri, utenti, prestiti) deve essere salvato su un unico file locale. Il salvataggio deve avvenire:\vspace{-6pt}
	\begin{itemize}
 	    \renewcommand{\labelitemi}{-}
		\setlength\itemsep{-0.1mm}
		\item{Su richiesta dell’utente.}
		\item{Automaticamente alla chiusura dell’applicazione.}
	\end{itemize}\vspace{-20pt} \phantom{X} }{Dati di archivio da salvare (quali l’insieme di studenti, libri e prestiti annessi). Il nome del file su cui devono essere salvati.}{File su sistema contenente i dati relativi all’archivio (archivio di utenti, libri e prestiti). Il modo in cui sono archiviati i file dipende dal requisito RNF-2.2.}{Nessuno.}{Nessuno.}{Alta}
\requisitebox{DF-2.2}{Caricamento da file dell’archivio}{Il sistema carica nelle proprie collezioni corrispondenti i libri, utenti e i prestiti presenti nel file di appoggio secondo la \\serializzazione utilizzata.}{Nome del file da cui caricare l’archivio.}{Strutture dati dell’archivio.}{File di appoggio presente e coerente.}{La memoria del sistema è popolata dalle informazioni presenti nel file di appoggio.}{Alta}

\newpage
\hypertarget{UI}{\subsubsection{Interfaccia Utente}}
\requisiteboxsimple{UI-1.1}{Menù principale dell’applicativo}{Landing page con menù di opzioni per accedere alle\\diverse funzionalità del sistema, aperto automaticamente all’avvio del\\programma grafico. Il menù mostra il nome del sistema e i diversi collegamenti per permettere al gestore di visualizzare i vari elenchi di informazioni, quali uno per visualizzare la lista degli utenti (UI-2.1), dei libri (UI-3.1), e dei prestiti attivi (UI-4.1).}{Alta}
\requisiteboxsimple{UI-1.2}{Interazioni persistenti}{Sezione di interazioni persistenti nella vista principale del programma. Contiene interazioni per:\vspace{-8pt}
	\begin{itemize}
		\renewcommand{\labelitemi}{-}
		\setlength\itemsep{0.1mm}
		\item{Salvare lo stato dell’archivio (nel file più recentemente associato, DF-2.1).}
		\item{Esportare l’archivio in un file specifico (DF-2.1).}
		\item{Caricare l’archivio da un file specifico (DF-2.2).}
		\item{Visualizzare nella pagina principale una delle pagine tra: l'elenco di utenti, l'elenco dei libri e l'elenco dei prestiti.}
	\end{itemize}\vspace{-8pt}
	L’interfaccia dell’applicativo viene aggiornata in base all'interazione scelta.
}{Media}
\requisiteboxsimple{UI-2.1}{Visualizzazione dell’elenco degli utenti}{Elenco grafico degli utenti registrati nell’archivio della biblioteca. La vista mostra le informazioni sugli utenti registrati\\nell’archivio della biblioteca includendo le loro informazioni (DF-1.1), \\ordinati alfabeticamente per cognome e, a parità di cognome, per nome, in una tabella. \\
Il sistema permette di modificare le informazioni relative all’utente direttamente dall'interfaccia,interagendo con la visualizzazione dei dati (IF-1.2). \\
Il sistema permette di filtrare la lista degli utenti tramite campi di cognome e matricola su richiesta (IF-1.4). \\
Il sistema permette di aggiungere utenti alla lista (UI-2.2) e di \\rimuovere l’utente selezionato (IF-1.3) su richiesta. \\
Dopo ogni operazione, la vista dell'elenco dovrà rispecchiare le \\modifiche apportate.}{Alta}
\requisiteboxsimple{UI-2.2}{Visualizzazione per la registrazione di un nuovo utente}{Vista per l’inserimento dei dati per la registrazione di un nuovo utente.\\
Il sistema permette l'inserimento delle informazioni necessarie (DF-1.1, eccetto la lista di libri in prestito), un'interazione per confermare \\l'operazione di registrazione al termine della compilazione delle \\informazioni (disponibile solo se le informazioni fornite in input \\rispettano le precondizioni del requisito IF-1.1) e un'interazione per annullare la registrazione del nuovo utente. La schermata per \\l'inserimento dell’utente viene chiusa al termine dell’operazione. Se la schermata viene chiusa con conferma della modifica da apportare, i dati relativi al nuovo utente saranno registrati nell’archivio.}{Alta}
\requisiteboxsimple{UI-3.1}{Visualizzazione dell’elenco dei libri}{Elenco grafico dei libri presenti nell’archivio della \\biblioteca. La vista mostra le informazioni sui libri presenti \\nell’archivio della biblioteca includendo le loro informazioni (DF-1.2), ordinati alfabeticamente per titolo, in una tabella.\\
Il sistema permette di modificare le informazioni relative al libro \\direttamente dall'interfaccia utente,interagendo con la visualizzazione dei dati (IF-2.2). Il sistema permette di filtrare la lista dei libri tramite campi di titolo, autore e codice ISBN su richiesta (IF-2.4). \\
Il sistema permette di aggiungere libri alla lista (UI-3.2), e di rimuovere il libro selezionato (IF-2.3) su richiesta. Dopo ogni operazione, la vista dell'elenco dovrà rispecchiare le modifiche apportate.}{Alta}
\requisiteboxsimple{UI-3.2}{Visualizzazione per la registrazione di un nuovo libro}{Vista per l’inserimento dei dati per la registrazione di un nuovo libro. Il sistema permette l'inserimento delle informazioni necessarie (DF-1.2), un interazione per confermare l'operazione di \\registrazione al termine della compilazione delle informazioni\\ (disponibile solo se le informazioni in input rispettano le precondizioni del requisito IF-2.1) e un interazione per annullare la registrazione del nuovo libro. La schermata per l'inserimento del libro viene chiusa al termine dell’operazione. Se la schermata viene chiusa con conferma della modifica da apportare, i dati relativi al nuovo libro saranno \\registrati nell’archivio.}{Alta}
\requisiteboxsimple{UI-4.1}{Visualizzazione dei prestiti attivi}{Elenco grafico dei prestiti attivi registrati nell’archivio della biblioteca. La vista mostra le informazioni sui prestiti registrati nell’archivio della biblioteca, includendo le loro informazioni (DF-1.3), ordinati in ordine crescente per data prevista di restituzione, in una tabella. Il sistema evidenzierà i prestiti attivi la cui data prevista di restituzione è precedente o uguale alla data di visualizzazione. \\
Il sistema permette di filtrare la lista dei prestiti per visualizzare \\solo i prestiti attivi o anche quelli restituititi. Il sistema permette di registrare un nuovo prestito (UI-4.2), e di segnare come restituito il prestito selezionato (BF-1.2) su richiesta. Dopo ogni operazione, la vista dell'elenco dovrà rispecchiare le modifiche apportate.}{Alta}
\requisiteboxsimple{UI-4.2}{Visualizzazione per la registrazione di un nuovo prestito}{Vista per l’inserimento dei dati per la registrazione di un nuovo prestito. Il sistema permette l'inserimento delle informazioni necessarie al prestito (DF-1.3), inserendo la matricola dell’utente e \\l'ISBN del libro (IF-1.1, IF-2.1) e la data prevista per la restituzione. \\
Il sistema permette un interazione per confermare l'operazione di \\registrazione al termine della compilazione delle informazioni \\(disponibile solo se le informazioni inserite rispettano le precondizioni del requisito BF-1.1) e un interazione per annullare la registrazione del nuovo prestito. La schermata per la registrazione del prestito viene chiusa al termine dell’operazione. Lo stato del nuovo prestito sarà \\automaticamente impostato come attivo. \\
Il sistema permette un interazione per mostrare la vista dello storico completo dei prestiti (per requisito UI-4.3). Se la schermata viene chiusa con conferma della modifica da apportare, i dati relativi al nuovo \\prestito saranno registrati nell’archivio.}{Alta}
\requisiteboxsimple{UI-4.3}{Visualizzazione dello storico dei prestiti}{Elenco grafico di tutti i prestiti, attivi e restituiti, \\registrati nell’archivio della biblioteca (IF-3.1). La visualizzazione si configura come per requisito UI-4.1, con la differenza che \\mostra nell'elenco grafico, oltre ai prestiti segnati come "attivi", anche i prestiti segnati come "restituiti". Il sistema permette di aggiungere nuovi prestiti e operare sui prestiti esistenti come per requisito UI-4.1.}{Bassa}

\hypertarget{RNF}{\subsubsection{Requisiti Non-Funzionali}}
\requisiteboxsimple{RNF-1.1}{Usabilità della piattaforma}{L’interfaccia grafica deve essere navigabile tramite mouse e tastiera, utilizzando pulsanti e elementi dinamici a schermo.}{Media}
\requisiteboxsimple{RNF-1.2}{Affidabilità della piattaforma}{Il sistema deve garantire che i dati siano sempre \\coerenti, impedendo duplicazioni o situazioni non valide come: copie disponibili negative, prestiti associati a utenti inesistenti, cancellazione di dati utilizzati da altre parti del sistema.}{Alta}

\newpage
\subsection{Tabella dei Requisiti}

{
	\footnotesize
	% Tabella dei requisiti
\definecolor{prioritygreen}{RGB}{106, 168, 79}    % Verde scuro
\definecolor{priorityyellow}{RGB}{241, 194, 50}   % Giallo scuro
\definecolor{priorityred}{RGB}{224, 102, 102}     % Rosso scuro
\definecolor{headercolor}{RGB}{79, 98, 120}       % Blu scuro
% Cella di identificativo della colonna
\newcommand{\headingcell}[1]{\multicolumn{1}{|c|}{#1}}
% Riga contenente le informazioni di un requisito
\newcommand{\requisiteline}[3] {
    \multicolumn{1}{c|}{\hyperlink{#1}{#1}} & #2 &
    \multicolumn{1}{c|}{\IfStrEqCase{#3}{
        {Bassa}{\cellcolor{prioritygreen}}
        {Media}{\cellcolor{priorityyellow}}
        {Alta}{\cellcolor{priorityred}}
    }[\cellcolor{white}]#3}\\
	\cline{2-4}
}

% Gestione dell'allineamento e dimensione delle colonne
\renewcommand\tabularxcolumn[1]{m{#1}}

\begin{xltabular}[h]{\textwidth}{
        | >{\rule{0pt}{30pt}\hsize=.4\hsize\centering\arraybackslash}X
        | >{\raggedright\hsize=.2\hsize}X
        | >{\raggedright\hsize=.3\hsize}X
        | >{\hsize=.1\hsize}X |
    }
	\hline
	\rowcolor{headercolor}\multicolumn{4}{|c|}{Lista dei Requisiti} \\
	\hline
    \rowcolor{lightgray}\headingcell{Categoria} & \headingcell{Identificativo} & \headingcell{Nome} & \headingcell{Priorità}\\
    \hline
    \multirow{3}{*}{\hyperlink{IF}{Funzionalità Individuali}} &
	    \requisiteline{IF-1.1}{Registrazione di un nuovo utente}{Alta} &
	    \requisiteline{IF-1.2}{Modifica dei dati di un utente}{Alta} &
	    \requisiteline{IF-1.3}{Rimozione di un utente}{Alta} &
	    \requisiteline{IF-1.4}{Ricerca di utenti}{Media} &
	    \requisiteline{IF-2.1}{Inserimento di un nuovo libro}{Alta} &
	    \requisiteline{IF-2.2}{Modifica dei dati relativi ad un libro}{Alta} &
	    \requisiteline{IF-2.3}{Rimozione di un libro}{Alta} &
	    \requisiteline{IF-2.4}{Ricerca di libri}{Media} &
	    \requisiteline{IF-3.1}{Raccolta dello storico dei prestiti}{Bassa} & 
	    \requisiteline{IF-3.2}{Modifica di un prestito}{Bassa}
    \hline
    \multirow{3}{*}{\hyperlink{BF}{Business Flow}} &
	    \requisiteline{BF-1.1}{Registrazione di un nuovo prestito}{Alta} &
	    \requisiteline{BF-1.2}{Restituzione di un libro}{Alta}
    \hline
    \multirow{3}{*}{\hyperlink{DF}{Esigenze Dati}} &
	    \requisiteline{DF-1.1}{Informazioni sull'utente}{Alta} &
	    \requisiteline{DF-1.2}{Informazioni sul libro}{Alta} &
	    \requisiteline{DF-1.3}{Informazioni sul prestito}{Alta} &
	    \requisiteline{DF-2.1}{Salvataggio dei dati dell’archivio su file}{Alta} &
	    \requisiteline{DF-2.2}{Caricamento da file dell’archivio}{Alta}
	\hline
	\multirow{3}{*}{\hyperlink{UI}{Interfaccia Utente}} &
		\requisiteline{UI-1.1}{Menù principale dell’applicativo}{Alta} &
	    \requisiteline{UI-1.2}{Menu-Bar persistente}{Media} &
	    \requisiteline{UI-2.1}{Visualizzazione dell’elenco degli utenti}{Alta} &
	    \requisiteline{UI-2.2}{Visualizzazione per la registrazione di un nuovo utente}{Alta} &
	    \requisiteline{UI-3.1}{Visualizzazione dell’elenco dei libri}{Alta} &
	    \requisiteline{UI-3.2}{Visualizzazione per la registrazione di un nuovo libro}{Alta} &
	    \requisiteline{UI-4.1}{Visualizzazione dei prestiti attivi}{Alta} &
	    \requisiteline{UI-4.2}{Visualizzazione per la registrazione di un nuovo prestito}{Alta} &
	    \requisiteline{UI-4.3}{Visualizzazione dello storico dei prestiti}{Bassa}
	\hline
	\multirow{3}{*}{\hyperlink{RNF}{Requisiti Non Funzionali}} &
		\requisiteline{RNF-1.1}{Usabilità della piattaforma}{Media} &
		\requisiteline{RNF-1.2}{Affidabilità della piattaforma}{Alta}
		%& \requisiteline{RNF-1.3}{Sicurezza dei dati}{Media}
		%& \requisiteline{RNF-1.4}{Scalabilità delle informazioni}{Bassa}
	\hline
\end{xltabular}

}

\newpage
\newcommand{\usecase}[7]{
	\begin{minipage}{\textwidth}
	\subsubsection{#2}
		\begin{description}
			\item\textbf{Attori partecipanti:}\\
			#3
			\item\textbf{Precondizioni:}\\
			#4
			\item\textbf{Postcondizioni:}\\
			#5
			\item\textbf{Flusso principale:}
			#6
			\item\textbf{Flussi alternativi:}
			#7
		\end{description}
	\end{minipage}
}

\section{Casi d'Uso}
\subsection{Descrizione dei Casi d'Uso}
\vspace{-2em}

\begin{description}[]\item[] \end{description}

\newcounter{ucid}
\renewcommand{\thesubsubsection}{UC-\arabic{subsubsection}}

\usecase{UC-\stepcounter{ucid}}{Visualizzazione della home page}{Gestore della biblioteca.}{\vspace{-0.6cm}}{L’home page viene visualizzata correttamente.}{
	\vspace{-0.8\topsep}
	\begin{enumerate}
		\item Il sistema carica correttamente l’archivio delle informazioni \\salvato in una precedente sessione.
		\item Il sistema mostra la visualizzazione dell’home page al gestore.
	\end{enumerate}
}{
	\vspace{-0.8\topsep}
	\begin{enumerate}[label=1A.\arabic*]
		\item[] \textbf{1A - L’archivio non è caricato correttamente}
		\item Il sistema mostra un messaggio di errore nella homepage.
		\item Il sistema genera un nuovo archivio vuoto da utilizzare.
	\end{enumerate}
}

\newpage

\usecase{UC-\stepcounter{ucid}}{Visualizzazione del catalogo dei libri}{Gestore della biblioteca.}{Nessuna.}{Il gestore ha consultato il catalogo dei libri.}{
	\vspace{-0.8\topsep}
	\begin{enumerate}
		\item Il gestore interagisce con l'opzione per visualizzare il catalogo.
		\item Il sistema mostra il catalogo aggiornato dei libri ordinato per titolo.
	\end{enumerate}
}{
	\vspace{-0.8\topsep}
	\begin{enumerate}[label=2A.\arabic*]
		\item[] \textbf{2A - Il gestore decide di ritornare alla home page}
		\item Il gestore seleziona l’opzione per tornare alla home page.
		\item Il sistema mostra la visualizzazione della home page come per requisito UC-1.
	\end{enumerate}
}

\vspace{0.5cm}
\hrule
\vspace{0.5cm}

\usecase{UC-\stepcounter{ucid}}{Visualizzazione dei prestiti attivi}{Gestore della biblioteca}{Nessuna.}{Le informazioni relative ai prestiti attivi vengono mostrate, evidenziando i prestiti in ritardo, in un elenco ordinato per data prevista di restituzione.}{
	\vspace{-0.8\topsep}
	\begin{enumerate}
		\item Il gestore interagisce con l’opzione per visualizzare i prestiti attivi.
		\item Il sistema mostra la lista dei prestiti attivi ordinata per data prevista di restituzione.
	\end{enumerate}
}{
	\vspace{-0.8\topsep}
	\begin{enumerate}[label=2A.\arabic*]
		\item[] \textbf{2A - Il gestore decide di ritornare alla home page}
		\item Il gestore seleziona l’opzione per tornare alla home page.
		\item Il sistema mostra la visualizzazione della home page come per requisito UC-1.
	\end{enumerate}
}

\vspace{0.5cm}
\hrule
\vspace{0.5cm}

\usecase{UC-\stepcounter{ucid}}{Visualizzazione dello storico dei prestiti}{Gestore della biblioteca}{Il gestore sta visualizzando la sezione relativa ai prestiti attivi.}{Le informazioni relative a tutti i prestiti vengono mostrate,evidenziando i prestiti in ritardo, in un elenco ordinato per data\\prevista di restituzione.}{
	\vspace{-0.8\topsep}
	\begin{enumerate}
		\item Il gestore interagisce con l'opzione per visualizzare anche i prestiti restituititi.
		\item Il sistema mostra la lista dei prestiti ordinata per data prevista di restituzione.
	\end{enumerate}
}{
	\vspace{-0.8\topsep}
	\begin{itemize}[label=]
		\item{Nessun flusso alternativo.}
	\end{itemize}
}
\newpage

\usecase{UC-\stepcounter{ucid}}{Visualizzazione dell’elenco degli utenti}{Gestore della biblioteca}{Nessuna.}{Viene mostrato l’elenco degli utenti ordinato alfabeticamente per\\cognome e, a parità di cognome, per nome.}{
	\vspace{-0.8\topsep}
	\begin{enumerate}
		\item Il gestore seleziona l’opzione per visualizzare l’elenco degli utenti.
		\item Il sistema mostra la lista degli utenti ordinata.
	\end{enumerate}
}{
	\vspace{-0.8\topsep}
	\begin{enumerate}[label=2A.\arabic*]
		\item[] \textbf{2A - Il gestore decide di ritornare alla home page}
		\item Il gestore seleziona l’opzione per tornare alla home page.
		\item Il sistema mostra la visualizzazione della home page come per requisito UC-1.
	\end{enumerate}
}

\newpage

\usecase{UC-\stepcounter{ucid}}{Registrazione di un nuovo utente}{Gestore della biblioteca}{Il gestore sta visualizzando l’elenco degli utenti.}{Il nuovo utente è registrato nell’archivio.}{
	\vspace{-0.8\topsep}
	\begin{enumerate}
		\item Il gestore interagisce con l'opzione per registrare un nuovo utente.
		\item Il sistema mostra al gestore la visualizzazione del form di\\registrazione per un nuovo utente.
		\item Il gestore inserisce i dati di nome, cognome, matricola ed email istituzionale del nuovo utente.
		\item Il sistema verifica la validità della matricola
		\item Il sistema verifica la validità dell'email.
		\item Il sistema verifica che la matricola non sia già utilizzata.
		\item Il gestore interagisce con l’opzione di conferma dell’operazione.
		\item Il sistema registra il nuovo utente.
	\end{enumerate}
}{	\vspace{-0.8\topsep}
\begin{enumerate}[label=4A.\arabic*]
	\item[] \textbf{4A - Matricola non valida, con correzioni}
	\item Il sistema rileva che la matricola non è valida
	\item Il gestore corregge la matricola inserendone una valida.
	\item L’esecuzione riprende dal passo 3.

\end{enumerate}
\phantom\\
\vspace{-0.8\topsep}
\begin{enumerate}[label=4B.\arabic*]
	\item[] \textbf{4B - Matricola non valida, con annullamento}
	\item Il sistema rileva che la matricola non è valida
	\item Il gestore interagisce con l’opzione per annullare l’operazione.
\end{enumerate}
\newpage
	\vspace{-0.8\topsep}
	\begin{enumerate}[label=5A.\arabic*]
		\item[] \textbf{5A - Email non valida, con correzione}
		\item Il sistema rileva che la e-mail inserita non è valida.
		\item Il sistema blocca l'interazione di conferma.
		\item Il gestore corregge l'e-mail.
		\item L’esecuzione riprende dal passo 6.
	\end{enumerate}

	\vspace{-0.8\topsep}
	\begin{enumerate}[label=5B.\arabic*]
		\item[] \textbf{5B - Email non valida, con annullamento}
		\item Il sistema rileva che la e-mail inserita non è valida.
		\item Il sistema blocca l'interazione di conferma.
		\item Il gestore interagisce con l’opzione per annullare l’operazione.

	\end{enumerate}

	\vspace{-0.8\topsep}
	\begin{enumerate}[label=6A.\arabic*]
		\item[] \textbf{6A - Matricola duplicata, con correzione}
		\item Il sistema rileva che la matricola inserita è già presente \\nell'archivio.
		\item Il sistema mostra un messaggio di errore.
		\item Il gestore corregge la matricola.
		\item L’esecuzione riprende dal passo 7.
	\end{enumerate}

	\vspace{-0.8\topsep}
	\begin{enumerate}[label=6B.\arabic*]
		\item[] \textbf{5B - Matricola duplicata, con annullamento}
		\item Il sistema rileva che la matricola inserita è già presente \\nell'archivio.
		\item Il sistema mostra un messaggio di errore.
		\item Il gestore interagisce con l’opzione per annullare l’operazione.
	\end{enumerate}
}

\newpage

\usecase{UC-\stepcounter{ucid}}{Modifica dei dati di un utente}{Gestore della biblioteca}{Il gestore sta visualizzando l'elenco degli utenti.}{Se valida: l’utente è aggiornato.\\In caso di errore: nessuna modifica applicata.}{
	\vspace{-0.8\topsep}
	\begin{enumerate}
		\item Il gestore individua l’utente di cui vuole modificare i dati.
		\item Il gestore seleziona il campo dell’utente che vuole modificare.
		\item Il gestore sovrascrive i dati nel campo con le nuove informazioni.
		\item Il sistema verifica la validità dei nuovi dati.
		\item Il sistema salva le modifiche.
	\end{enumerate}
}{
	\vspace{-0.8\topsep}
	\begin{enumerate}[label=4A.\arabic*]
		\item[] \textbf{4A - Nuova E-Mail non valida}
		\item Il sistema rileva l'inserimento di una e-mail non conforme.
		\item Il sistema annulla la modifica e non modifica i dati nell'archivio.
	\end{enumerate}
}

\newpage

\usecase{UC-\stepcounter{ucid}}{Rimozione di un utente}{Gestore della biblioteca}{Il gestore sta visualizzando l'elenco degli utenti.}{L’utente viene rimosso dall'archivio.}{
	\vspace{-0.8\topsep}
	\begin{enumerate}
		\item Il gestore individua e seleziona l'utente da rimuovere.
		\item Il sistema verifica l’assenza di prestiti attivi.
		\item Il gestore interagisce con l'opzione di rimozione dell'utente.
		\item Il sistema rimove l’utente dall'archivio.
	\end{enumerate}
}{
	\vspace{-0.8\topsep}
	\begin{enumerate}[label=2A.\arabic*]
		\item[] \textbf{2A - L'utente possiede prestiti attivi}
		\item Il sistema disabilita l'opzione di rimozione per quell'utente.
	\end{enumerate}
}

\newpage

\usecase{UC-\stepcounter{ucid}}{Ricerca di utenti}{Gestore della biblioteca}{Il gestore sta visualizzando l'elenco degli utenti.}{Viene mostrato l’elenco filtrato.}{
	\vspace{-0.8\topsep}
	\begin{enumerate}
		\item Il gestore inserisce i campi di cognome e matricola nel form per la ricerca.
		\item Il sistema mostra nell'elenco solo gli utenti che corrispondono ai criteri di ricerca.
	\end{enumerate}
}{
	\vspace{-0.8\topsep}
	\begin{itemize}[label=]
		\item{Nessun flusso alternativo.}
	\end{itemize}
}

\newpage
\usecase{UC-\stepcounter{ucid}}{Registrazione di un prestito}{Gestore della biblioteca}{Il gestore sta visualizzando l'elenco dei prestiti attivi.}{Prestito registrato.\\Copie disponibili del libro prestato decrementate.\\Prestito aggiunto alla lista dei prestiti dell'utente.}{
	\vspace{-0.8\topsep}
	\begin{enumerate}
		\item Il gestore seleziona l'opzione per aggiungere un prestito.
		\item Il gestore inserisce la matricola dell’utente a carico del prestito e l'ISBN del libro da prestare.
		\item Il gestore inserisce la data di restituzione prevista per il prestito.
		\item Il sistema verifica disponibilità di copie del libro in prestito, che il limite di prestiti dell’utente non sia stato raggiunto e che la data di scadenza sia successiva a quella di registrazione.
		\item Il gestore seleziona l'opzione per confermare l'operazione.
		\item Il prestito viene registrato nell'archivio del sistema.
	\end{enumerate}
}{
	\vspace{-0.8\topsep}
	\begin{enumerate}[label=4A.\arabic*]
		\item[] \textbf{4A - Limite di prestiti raggiunto, con correzione}
		\item Se l'utente ha già 3 prestiti attivi, il sistema non permette al gestore di confermare l'operazione.
		\item Il gestore corregge i dati inseriti cambiando l'utente a carico.
		\item Il flusso riprende dal passo 5.
	\end{enumerate}

	\vspace{-0.8\topsep}
	\begin{enumerate}[label=4B.\arabic*]
		\item[] \textbf{4B - Limite di prestiti raggiunto, senza correzione}
		\item Se l'utente ha già 3 prestiti attivi, il sistema non permette al gestore di confermare l'operazione.
		\item Il gestore annulla l'operazione.
		\item Il prestito non viene registrato nell'archivio del sistema.
	\end{enumerate}
	\newpage
	\vspace{-0.8\topsep}
	\begin{enumerate}[label=4C.\arabic*]
		\item[] \textbf{4C - Copie del libro terminate, con correzione}
		\item Se non ci sono copie disponibili del libro da prestare nell'archivio, il sistema non permette al gestore di confermare l'operazione.
		\item Il gestore corregge i dati inseriti cambiando il libro da prestare.
		\item Il flusso riprende dal passo 5.
	\end{enumerate}

	\vspace{-0.8\topsep}
	\begin{enumerate}[label=4D.\arabic*]
		\item[] \textbf{4D - Copie del libro terminate, senza correzione}
		\item Se non ci sono copie disponibili del libro da prestare nell'archivio, il sistema non permette al gestore di confermare l'operazione.
		\item Il gestore annulla l'operazione.
		\item Il prestito non viene registrato nell'archivio del sistema.
	\end{enumerate}

	\vspace{-0.8\topsep}
	\begin{enumerate}[label=4E.\arabic*]
		\item[] \textbf{4E - Data di scadenza nel passato, senza correzione}
		\item Se la data di scadenza è precedente a quella di registrazione, il sistema non
		permette al gestore di confermare l'operazione.
		\item Il gestore annulla l'operazione.
		\item Il prestito non viene registrato nell'archivio del sistema.
	\end{enumerate}

	\vspace{-0.8\topsep}
	\begin{enumerate}[label=4F.\arabic*]
		\item[] \textbf{4F - Data di scadenza nel passato, con correzione}
		\item Se la data di scadenza è precedente a quella di registrazione, il sistema non
		permette al gestore di confermare l'operazione.
		\item Il gestore corregge i dati inseriti inserendo una data di scadenza valida.
		\item Il flusso riprende dal passo 5.
	\end{enumerate}

	\vspace{-0.8\topsep}
	\begin{enumerate}[label=5A.\arabic*]
		\item[] \textbf{5A - Annullamento dell'operazione}
		\item Il gestore annulla l'operazione.
		\item Il prestito non viene registrato nell'archivio del sistema.
	\end{enumerate}
}

\newpage

\usecase{UC-\stepcounter{ucid}}{Restituzione di un libro}{Gestore della biblioteca}{Il gestore sta visualizzando l'elenco dello storico dei prestiti.}{Il prestito è segnato come restituito.\\Il numero di copie disponibili del libro restituito è incrementato.\\Prestito rimosso da lista prestiti utente}{
	\vspace{-0.8\topsep}
	\begin{enumerate}
		\item Il gestore seleziona il prestito da restituire dall'elenco.
		\item Il gestore interagisce con l'opzione per confermare la restituzione del libro.
		\item Il prestito viene segnato come restituito.
		\item Il sistema aggiorna il numero di copie del libro restituito.
		\item Il sistema aggiorna la lista dei prestiti attivi dell'utente
	\end{enumerate}
}{
	\vspace{-0.8\topsep}
	\begin{enumerate}[label=2A.\arabic*]
		\item[] \textbf{2A - Annullamento dell'operazione}
		\item Il gestore deseleziona il prestito da restituire.
		\item Il sistema non modifica lo stato del prestito.
	\end{enumerate}
}

\newpage

\usecase{UC-\stepcounter{ucid}}{Ricerca di libri}{Gestore della biblioteca}{Il gestore sta visualizzando l'elenco dei libri.}{Viene mostrato l’elenco filtrato.}{
	\vspace{-0.8\topsep}
	\begin{enumerate}
		\item Il gestore inserisce i campi di titolo, autore o ISBN nel form per la ricerca.
		\item Il gestore interagisce con l'opzione per filtrare i risultati a seconda dei campi inseriti.
		\item Il sistema mostra nell'elenco solo i libri che corrispondono ai criteri di ricerca.
	\end{enumerate}
}{
	\vspace{-0.8\topsep}
	\begin{itemize}[label=]
		\item{Nessun flusso alternativo.}
	\end{itemize}
}

\newpage

\usecase{UC-\stepcounter{ucid}}{Inserimento di un nuovo libro}{Gestore della biblioteca}{Il gestore sta visualizzando l'elenco dei libri.}{Il nuovo libro è registrato nell’archivio.}{
	\vspace{-0.8\topsep}
	\begin{enumerate}
		\item Il gestore interagisce con l'opzione di aggiunta di un nuovo libro.
		\item Il sistema mostra al gestore la visualizzazione del form di\\registrazione per un nuovo libro.
		\item Il gestore inserisce nei campi idonei titolo, lista di autori, data di pubblicazione, ISBN e numero copie disponibili nella biblioteca.
		\item Il sistema verifica la validità dell'ISBN.
		\item Il sistema verifica l'unicità dell’ISBN nell'archivio.
		\item Il gestore interagisce con l'opzione per confermare l'operazione.
		\item Il sistema registra il libro e conferma l’inserimento.
	\end{enumerate}
}{
	\vspace{-0.8\topsep}
	\begin{enumerate}[label=4A.\arabic*]
		\item[] \textbf{4A - ISBN non valido, con correzioni}
		\item Il sistema rileva un ISBN non conforme.
		\item Il gestore corregge l'ISBN inserendone uno valido
		\item Il flusso riprende dal passo 5.
	\end{enumerate}

	\begin{enumerate}[label=4B.\arabic*]
		\item[] \textbf{4B - ISBN non valido, con annullamento}
		\item Il sistema rileva che la matricola non è valida
		\item Il gestore interagisce con l’opzione per annullare l’operazione.
	\end{enumerate}

	\vspace{-0.8\topsep}
	\begin{enumerate}[label=5A.\arabic*]
		\item[] \textbf{5A - ISBN già presente nell'archivio, con correzioni}
		\item Il sistema rileva che l'ISBN inserito è già presente nell'archivio.
		\item Il sistema mostra un messaggio di errore.
		\item Il gestore corregge l'ISBN.
		\item Il flusso riprende dal passo 6.
	\end{enumerate}

	\vspace{-0.8\topsep}
	\begin{enumerate}[label=5B.\arabic*]
		\item[] \textbf{5B - ISBN già presente nell'archivio, con annullamento}
		\item Il sistema rileva che l'ISBN inserito è già presente nell'archivio.
		\item Il sistema mostra un messaggio di errore.
		\item Il gestore interagisce con l’opzione per annullare l’operazione.
	\end{enumerate}

	\vspace{-0.8\topsep}
	\begin{enumerate}[label=6A.\arabic*]
		\item[] \textbf{6A - Annullamento dell'operazione}
		\item Il gestore decide di non confermare l'operazione e interagisce con l'opzione di annullamento.
		\item Il sistema non registra la registrazione del nuovo libro.
	\end{enumerate}
}

\newpage

\usecase{UC-\stepcounter{ucid}}{Modifica dei dati di un libro}{Gestore della biblioteca}{Il gestore sta visualizzando l'elenco dei libri.}{I dati sul libro sono correttamente aggiornati.}{
	\vspace{-0.8\topsep}
	\begin{enumerate}
		\item Il gestore individua il libro da modificare nell'elenco.
		\item Il gestore seleziona il campo del libro da modificare.
		\item Il gestore sovrascrive il campo con il nuovo dato.
		\item Il sistema verifica la validità dei dati inseriti.
		\item Il sistema salva le modifiche.
	\end{enumerate}
}{
	\vspace{-0.8\topsep}
	\begin{enumerate}[label=4A.\arabic*]
		\item[] \textbf{4A - Inserimento data di pubbicazione non valida, con correzioni}
		\item Il sistema rileva che la data di pubblicazione inserita è successiva a quella attuale.
		\item Il sistema mostra un messaggio di errore.
		\item Il gestore modifica la data con una corretta.
		\item Il flusso riprende dal passo 5.
	\end{enumerate}

	\vspace{-0.8\topsep}
	\begin{enumerate}[label=4B.\arabic*]
		\item[] \textbf{4B - Inserimento data di pubbicazione non valida, con annullamento}
		\item Il sistema rileva che la data di pubblicazione inserita è successiva a quella attuale.
		\item Il sistema mostra un messaggio di errore.
		\item Il sistema rifiuta la modifica e il dato resta invariato.
	\end{enumerate}

	\newpage

	\vspace{-0.8\topsep}
	\begin{enumerate}[label=4C.\arabic*]
		\item[] \textbf{4C - Inserimento vuoto campo titolo o autore, con correzioni}
		\item Il sistema rileva che il titolo o autore è vuoto.
		\item Il sistema mostra un messaggio di errore.
		\item Il gestore modifica i campi inserendo titolo o autore non vuoto.
		\item Il flusso riprende dal passo 5.
	\end{enumerate}

	\vspace{-0.8\topsep}
	\begin{enumerate}[label=4D.\arabic*]
		\item[] \textbf{4D - Inserimento data di pubbicazione non valida, con annullamento}
		\item Il sistema rileva che il titolo o autore è vuoto.
		\item Il sistema mostra un messaggio di errore.
		\item Il sistema rifiuta la modifica e il dato resta invariato.
	\end{enumerate}
}

\newpage

\usecase{UC-\stepcounter{ucid}}{Rimozione di un libro dal catalogo}{Gestore della biblioteca}{Il gestore sta visualizzando l'elenco dei libri.}{Il libro viene rimosso.}{
	\vspace{-0.8\topsep}
	\begin{enumerate}
		\item Il gestore seleziona il libro da rimuovere.
		\item Il gestore interagisce con l'opzione di rimozione dell'interfaccia.
		\item Il libro viene rimosso dall'archivio.
		\item L’elenco dei libri viene aggiornato.
	\end{enumerate}
}{
	\vspace{-0.8\topsep}
	\begin{itemize}[label=]
		\item{Nessun flusso alternativo.}
	\end{itemize}
}


\newpage
\usecase{UC-\stepcounter{ucid}}{Modifica di un prestito}{Gestore della biblioteca.}{Il gestore sta visualizzando l'elenco dei prestiti.}{La data di scadenza del prestito è stata modificata.}{
	\vspace{-0.8\topsep}
	\begin{enumerate}
		\item Il gestore individua il prestito da modificare.
		\item Il gestore seleziona la data di scadenza del prestito.
		\item Il gestore inserisce la nuova data di restituzione.
		\item Il sistema verifica la validità della nuova data inserita.
		\item Il sistema salva le modifiche.
	\end{enumerate}
}{
	\vspace{-0.8\topsep}
	\begin{enumerate}[label=4A.\arabic*]
		\item[] \textbf{4A - Data di scadenza non valida, con correzione}
		\item Il sistema rileva che la nuova data di restituzione è precedente a quella di registrazione.
		\item Il sistema mostra un messaggio di errore.
		\item Il gestore modifica il campo inserendo una data valida.
		\item Il flusso riprende dal passo 5.
	\end{enumerate}

	\vspace{-0.8\topsep}
	\begin{enumerate}[label=4B.\arabic*]
		\item[] \textbf{4B - Data di scadenza non valida, con correzione}
		\item Il sistema rileva che la nuova data di restituzione è precedente a quella di registrazione.
		\item Il sistema mostra un messaggio di errore.
		\item la modifca non viene accettata e il sistema torna alla data\\ precedente.
	\end{enumerate}
}

\newpage

\usecase{UC-\stepcounter{ucid}}{Salvataggio dell'archivio su file}{Gestore della biblioteca.}{L'archivio del sistema è caricato in memoria.}{Il file di appoggio specificato è riempito delle informazioni \\dell'archivio}{
	\vspace{-0.8\topsep}
	\begin{enumerate}
		\item Il gestore interagisce con l'opzione per salvare l'archivio su file.
		\item Il gestore indica al sistema su quali file salvare l'archivio.
		\item Il sistema controlla che sia possibile accedere e scrivere sul file specificati.
		\item Il sistema scrive sul file i dati dell'archivio.
	\end{enumerate}
}{
	\vspace{-0.8\topsep}
	\begin{enumerate}[label=3A.\arabic*]
		\item[] \textbf{3A - Errore di input/output sul filesystem}
		\item Il sistema rileva l'impossibilità a salvare i dati sul file specificati.
		\item Il sistema mostra un messaggio di errore e annulla l'operazione.
	\end{enumerate}
}


\newpage

\usecase{UC-\stepcounter{ucid}}{Caricamento dell'archivio dal file}{Gestore della biblioteca.}{Il gestore ha visualizzato l'home page.}{L'archivio è popolato dalle informazioni presenti nel file di appoggio.}{
	\vspace{-0.8\topsep}
	\begin{enumerate}
		\item Il gestore interagisce con l'opzione per caricare l'archivio dal file.
		\item Il gestore indica al sistema da quali file caricare l'archivio.
		\item Il sistema controlla che sia possibile accedere e leggere dal file specificato.
		\item Il sistema popola l'archivio con i dati letti dal file, sovrascrivendo i dati precedentemente memorizzati.
	\end{enumerate}
}{
	\vspace{-0.8\topsep}
	\begin{enumerate}[label=3A.\arabic*]
		\item[] \textbf{3A - Errore di input/output sul filesystem}
		\item Il sistema rileva l'impossibilità a leggere i dati dal file specificati.
		\item Il sistema mostra un messaggio di errore e annulla l'operazione.
	\end{enumerate}

	\vspace{-0.8\topsep}
	\begin{enumerate}[label=4A.\arabic*]
		\item[] \textbf{4A - Errore di interpretazione dei dati}
		\item Il sistema rileva che i dati presenti nel file specificato non\\corrispondono ad un corretto archivio del sistema da cui poter estrarre informazioni.
		\item Il sistema mostra un messaggio di errore e annulla l'operazione.
	\end{enumerate}
}

\renewcommand{\thesubsubsection}{\arabic{section}.\arabic{subsection}.\arabic{subsubsection}}

\newpage
\subsection{Diagrammi UML dei Casi d'Uso}
\subsubsection{Casi d'Uso per la gestione di Utenti}
\begin{figure}[ht]
\begin{adjustbox}{center, margin=0cm 0cm 0cm 0cm}
	\includesvg[width=1.2\textwidth, inkscapelatex=false]{../uml/usecasesUtente.svg}
\end{adjustbox}
\end{figure}

\newpage
\subsubsection{Casi d'Uso per la gestione di Libri}
\begin{figure}[ht]
	\begin{adjustbox}{center, margin=0cm 0cm 0cm 0cm}
		\includesvg[width=1.2\textwidth, inkscapelatex=false]{../uml/usecasesLibro.svg}
	\end{adjustbox}
\end{figure}

\newpage
\subsubsection{Casi d'Uso per la gestione di Prestiti}
\begin{figure}[ht]
	\begin{adjustbox}{center, margin=0cm 0cm 0cm 0cm}
		\includesvg[width=1.2\textwidth, inkscapelatex=false]{../uml/usecasesPrestito.svg}
	\end{adjustbox}
\end{figure}

\newpage
\subsubsection{Casi d'Uso per l'archiviazione}
\begin{figure}[ht]
	\begin{adjustbox}{center, margin=0cm 0cm 0cm 0cm}
		\includesvg[width=1.2\textwidth, inkscapelatex=false]{../uml/usecasesIO.svg}
	\end{adjustbox}
\end{figure}

\end{document}
