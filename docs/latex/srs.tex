\documentclass[a4paper, 12pt]{article}
\usepackage[italian]{babel}
\usepackage[table]{xcolor}
\usepackage{xstring, xltabular, multirow, ifthen, hyphenat, tcolorbox}
\usepackage[
	colorlinks = true,
	linkcolor=black,
	urlcolor=black,
	citecolor=black
]{hyperref}


\hypersetup{linkcolor=black}

% Titolo del documento e informazioni sul lavoro
\title{
    \Large EasyLibrary\\
    \vspace{1mm}
    \Huge Specifica dei Requisiti del Software
}
% Metodo per centrare correttamente con wrapping i nomi.
\author{
    \textbf{Gruppo 18}\\
    Francesco Cangianiello\\
    Andrea Ciliberti\\
    Serena Giannitti\\
    Marco Giraulo
}

\begin{document}

\maketitle

\newpage
\tableofcontents

\newpage
\section{Requisiti di Progetto}
\subsection{Lista dei Requisiti}

% Scope della tabella dei requisiti
{
	% Tabella dei requisiti
\definecolor{prioritygreen}{RGB}{106, 168, 79}    % Verde scuro
\definecolor{priorityyellow}{RGB}{241, 194, 50}   % Giallo scuro
\definecolor{priorityred}{RGB}{224, 102, 102}     % Rosso scuro
\definecolor{headercolor}{RGB}{79, 98, 120}       % Blu scuro
% Cella di identificativo della colonna
\newcommand{\headingcell}[1]{\multicolumn{1}{|c|}{#1}}
% Riga contenente le informazioni di un requisito
\newcommand{\requisiteline}[3] {
    \multicolumn{1}{c|}{\hyperlink{#1}{#1}} & #2 &
    \multicolumn{1}{c|}{\IfStrEqCase{#3}{
        {Bassa}{\cellcolor{prioritygreen}}
        {Media}{\cellcolor{priorityyellow}}
        {Alta}{\cellcolor{priorityred}}
    }[\cellcolor{white}]#3}\\
	\cline{2-4}
}

% Gestione dell'allineamento e dimensione delle colonne
\renewcommand\tabularxcolumn[1]{m{#1}}

\begin{xltabular}[h]{\textwidth}{
        | >{\rule{0pt}{30pt}\hsize=.4\hsize\centering\arraybackslash}X
        | >{\raggedright\hsize=.2\hsize}X
        | >{\raggedright\hsize=.3\hsize}X
        | >{\hsize=.1\hsize}X |
    }
	\hline
	\rowcolor{headercolor}\multicolumn{4}{|c|}{Lista dei Requisiti} \\
	\hline
    \rowcolor{lightgray}\headingcell{Categoria} & \headingcell{Identificativo} & \headingcell{Nome} & \headingcell{Priorità}\\
    \hline
    \multirow{3}{*}{\hyperlink{IF}{Funzionalità Individuali}} &
	    \requisiteline{IF-1.1}{Registrazione di un nuovo utente}{Alta} &
	    \requisiteline{IF-1.2}{Modifica dei dati di un utente}{Alta} &
	    \requisiteline{IF-1.3}{Rimozione di un utente}{Alta} &
	    \requisiteline{IF-1.4}{Ricerca di utenti}{Media} &
	    \requisiteline{IF-2.1}{Inserimento di un nuovo libro}{Alta} &
	    \requisiteline{IF-2.2}{Modifica dei dati relativi ad un libro}{Alta} &
	    \requisiteline{IF-2.3}{Rimozione di un libro}{Alta} &
	    \requisiteline{IF-2.4}{Ricerca di libri}{Media} &
	    \requisiteline{IF-3.1}{Raccolta dello storico dei prestiti}{Bassa} & 
	    \requisiteline{IF-3.2}{Modifica di un prestito}{Bassa}
    \hline
    \multirow{3}{*}{\hyperlink{BF}{Business Flow}} &
	    \requisiteline{BF-1.1}{Registrazione di un nuovo prestito}{Alta} &
	    \requisiteline{BF-1.2}{Restituzione di un libro}{Alta}
    \hline
    \multirow{3}{*}{\hyperlink{DF}{Esigenze Dati}} &
	    \requisiteline{DF-1.1}{Informazioni sull'utente}{Alta} &
	    \requisiteline{DF-1.2}{Informazioni sul libro}{Alta} &
	    \requisiteline{DF-1.3}{Informazioni sul prestito}{Alta} &
	    \requisiteline{DF-2.1}{Salvataggio dei dati dell’archivio su file}{Alta} &
	    \requisiteline{DF-2.2}{Caricamento da file dell’archivio}{Alta}
	\hline
	\multirow{3}{*}{\hyperlink{UI}{Interfaccia Utente}} &
		\requisiteline{UI-1.1}{Menù principale dell’applicativo}{Alta} &
	    \requisiteline{UI-1.2}{Menu-Bar persistente}{Media} &
	    \requisiteline{UI-2.1}{Visualizzazione dell’elenco degli utenti}{Alta} &
	    \requisiteline{UI-2.2}{Visualizzazione per la registrazione di un nuovo utente}{Alta} &
	    \requisiteline{UI-3.1}{Visualizzazione dell’elenco dei libri}{Alta} &
	    \requisiteline{UI-3.2}{Visualizzazione per la registrazione di un nuovo libro}{Alta} &
	    \requisiteline{UI-4.1}{Visualizzazione dei prestiti attivi}{Alta} &
	    \requisiteline{UI-4.2}{Visualizzazione per la registrazione di un nuovo prestito}{Alta} &
	    \requisiteline{UI-4.3}{Visualizzazione dello storico dei prestiti}{Bassa}
	\hline
	\multirow{3}{*}{\hyperlink{RNF}{Requisiti Non Funzionali}} &
		\requisiteline{RNF-1.1}{Usabilità della piattaforma}{Media} &
		\requisiteline{RNF-1.2}{Affidabilità della piattaforma}{Alta}
		%& \requisiteline{RNF-1.3}{Sicurezza dei dati}{Media}
		%& \requisiteline{RNF-1.4}{Scalabilità delle informazioni}{Bassa}
	\hline
\end{xltabular}

}

\newpage

\newcommand{\requisitebox}[8]{%
	\hypertarget{#1}{}
	\begin{tcolorbox}[colback=white,colframe=black!60,rounded corners]
		\textbf{ID:} #1 \\[2pt]
		\textbf{Nome:} #2 \\[2pt]
		\textbf{Descrizione:} #3 \\[2pt]
		\textbf{Input:} #4 \\[2pt]
		\textbf{Output:} #5 \\[2pt]
		\textbf{Precondizioni:} #6 \\[2pt]
		\textbf{Postcondizioni:} #7 \\[2pt]
		\textbf{Priorità:} #8
	\end{tcolorbox}
}

\subsection{Specifica dei Requisiti}
\hypertarget{IF}{\subsubsection{Funzionalità individuali}}
\requisitebox{DF-1}{bingo}{Bango}{Nessuno.}{Nessuno.}{Nessuno.}{Nessuno.}{Alta}
\hypertarget{BF}{\subsubsection{Business Flow}}
\hypertarget{DF}{\subsubsection{Esigenze Dati}}
\hypertarget{UI}{\subsubsection{Interfaccia Utente}}
\hypertarget{RNF}{\subsubsection{Requisiti Non-Funzionali}}

\newpage
\section{Casi d'Uso}
\subsection{Descrizione dei Casi d'Uso}
\subsection{Diagramma UML dei Casi d'Uso}

\end{document}
